
\documentclass[12pt]{article}

\usepackage{amsmath}
\usepackage{graphicx}
\usepackage{hyperref}
\usepackage[utf8]{inputenc}
\usepackage[T5]{fontenc}
\usepackage{fancyhdr}
\usepackage{geometry}

\usepackage{csquotes}
\usepackage{listings}
\usepackage{enumitem}
\usepackage{subfiles}
\usepackage{tcolorbox}
\usepackage{colortbl}
\usepackage{float}
\usepackage{circuitikz}
\usepackage{tikz}

% FORMATTING

\geometry{
    a4paper,
    total={170mm,250mm},
    left=20mm,
    top=30mm,
 }
\hypersetup{
    colorlinks=true,
    linkcolor=blue,
    filecolor=magenta,      
    urlcolor=blue,
    citecolor=blue
}

\pagestyle{fancy}
\fancyhf{}
\rhead{19CTT4}
\lhead{Mạch Logic}
\rfoot{Trang \thepage}\setlength{\parindent}{0pt}
\setlength{\parskip}{0.3em}
\setlength{\headheight}{15pt}
\graphicspath{ {./images/} }

% DEFINE NEW COMMAND
\newcommand{\SubItem}[1]{
    {\setlength\itemindent{15pt} \item[-] #1}
}

% DOCUMENT
\title{PHẦN 4: MẠCH LOGIC}

\author{LỚP 19CTT4}

\date{2021–07–05}


\begin{document}

\begin{titlepage}
    
    \newcommand{\HRule}{\rule{\linewidth}{0.5mm}} % Defines a new command for the horizontal lines, change thickness here
    
    \center % Center everything on the page
    \vspace*{\fill}
     
    \textsc{\LARGE Đại học Khoa học Tự nhiên}\\[0.2cm]
    \textsc{\large Đại học Quốc gia TP. HCM }\\[1.5cm] 
    \textsc{\Large KHOA CÔNG NGHỆ THÔNG TIN}\\[0.2cm] 
    \textsc{\large LỚP 19CTT4 }\\[0.5cm]
    \HRule \\[0.4cm]
    { \huge \bfseries Tài liệu ôn thi cuối kỳ môn\break 
    Hệ thống máy tính}\\[0.4cm] % Title of your document
    \HRule \\[1.5cm]
    \LARGE \textbf {PHẦN 4: MẠCH LOGIC \\}
    
    \begin{minipage}{1\textwidth}
    \begin{center}
        \LARGE Ngày 05/07/2021
    \end{center}
    \end{minipage}\\[2cm]
    \includegraphics[width=10em]{images/profile.png}
    \vspace*{\fill} % Fill the rest of the page with whitespace
    \end{titlepage}


    % MỤC LỤC
    \renewcommand*\contentsname{\begin{center} \LARGE Mục lục \end{center}}
    \setcounter{tocdepth}{2}
    \tableofcontents
    \pagebreak
\section{Khái niệm mạch số}

Là thiết bị điện tử hoạt động với \textcolor{red}{2 mức điện áp:}

\begin{itemize}

    \item \textcolor{red}{Cao:} thể hiện bằng giá trị luận lý (quy ước) là \textcolor{red}{1}.
    \item \textcolor{red}{Thấp:} thể hiện bằng giá trị luận lý (quy ước) là \textcolor{red}{0}. 
\end{itemize}


Được xây dựng từ những thành phần cơ bản là \textcolor{red}{cổng luận lý (logic gate)}
\begin{itemize}
    \item Cổng luận lý là thiết bị điện tử gồm 1/ nhiều tín hiệu đầu vào (input) - 1 tín hiệu đầu ra output.
    \item \textcolor{red}{\begin{math}output = F (input\_1, input\_2, ..., input\_n)\end{math}.}
    \item Tùy thuộc vào cách xử lý của hàm F sẽ tạo ra nhiều loại cổng luận lý.
\end{itemize}

Hiện nay linh kiện cơ bản tạo ra mạch số là \textcolor{red}{transistor}.

\subsection{Cổng luận lý(Logic gate)}



\begin{table}[H]
    \centering
    % \begin{tabular}{|l|l|l|}
    \resizebox{\textwidth}{!}{\begin{tabular}{|l|c|l|}
    \hline
    \multicolumn{1}{|c|}{{\color[HTML]{9A0000} \textbf{Tên cổng}}} & \multicolumn{1}{c|}{{\color[HTML]{9A0000} \textbf{Hình vẽ đại diện}}}                       & \multicolumn{1}{c|}{{\color[HTML]{9A0000} \textbf{Hàm đại số Bun}}} \\ \hline
    {\color[HTML]{963400} AND}                                     & \begin{circuitikz} \draw (4,4) node[and port, scale = 0.5] {}; \end{circuitikz}    & \textbackslash{}x.y hay \textbackslash{}xy                          \\ \hline
    {\color[HTML]{963400} OR}                                      & \begin{circuitikz} \draw (4,4) node[or port, scale = 0.5] {}; \end{circuitikz}     & x + Y                                                               \\ \hline
    {\color[HTML]{963400} XOR}                                     & \begin{circuitikz} \draw (0,1) node[xor port, scale = 0.5] {}; \end{circuitikz}    & \textbackslash{}oplus                                               \\ \hline
    {\color[HTML]{963400} NOT}                                     & \begin{circuitikz} \draw (0,1) node[not port, scale = 0.5] {}; \end{circuitikz}    &                                                                     \\ \hline
    {\color[HTML]{963400} NAND}                                    & \begin{circuitikz} \draw (0,1) node[nand port, scale = 0.5] {}; \end{circuitikz}   &                                                                     \\ \hline
    {\color[HTML]{963400} NOR}                                     & \begin{circuitikz} \draw (0,1) node[nor port, scale = 0.5] {}; \end{circuitikz}    &                                                                     \\ \hline
    {\color[HTML]{963400} NXOR}                                    & \begin{circuitikz} \draw (0,1) node[nor port, scale = 0.5] {}; \end{circuitikz}   &                                                                     \\ \hline
    \end{tabular}}
    \end{table}


\subsection{Bảng chân trị}

\begin{tcolorbox}
    \textbf{Example:} This is a box    
\end{tcolorbox}

\begin{itemize}
    \item \textbf{\textcolor{red}{AND}}
    
    \begin{circuitikz} \draw
        (0,1) node[and port] (myand1) {}
            (myand1.in 1) node [anchor=east] {A}
            (myand1.in 2) node [anchor=east] {B}
            (myand1.out)  node [anchor=west] {out};

        \end{circuitikz}
    \begin{table}[H]
        \centering
        \begin{tabular}{|c|c|
        >{\columncolor[HTML]{F8FF00}}l |}
        \hline
        \cellcolor[HTML]{34CDF9}A & \cellcolor[HTML]{34CDF9}B & out                      \\ \hline
        0                         & 0                         & 0                        \\ \hline
        0                         & 1                         & 0                        \\ \hline
        1                         & 0                         & 0                        \\ \hline
        {\color[HTML]{FE0000} 1}  & {\color[HTML]{FE0000} 1}  & {\color[HTML]{FE0000} 1} \\ \hline
        \end{tabular}
        \end{table}
    \item \textbf{\textcolor{red}{OR}}
    
    \begin{circuitikz} \draw
        (0,1) node[or port] (myor1) {}
            (myor1.in 1) node [anchor=east] {A}
            (myor1.in 2) node [anchor=east] {B}
            (myor1.out)  node [anchor=west] {out};

        \end{circuitikz}

    \begin{table}[H]
        \centering
        \begin{tabular}{|c|c|
        >{\columncolor[HTML]{F8FF00}}l |}
        \hline
        \cellcolor[HTML]{34CDF9}A & \cellcolor[HTML]{34CDF9}B & out                      \\ \hline
        {\color[HTML]{FE0000} 0}  & {\color[HTML]{FE0000} 0}  & {\color[HTML]{FE0000} 0} \\ \hline
        0                         & 1                         & 1                        \\ \hline
        1                         & 0                         & 1                        \\ \hline
        {\color[HTML]{333333} 1}  & {\color[HTML]{333333} 1}  & {\color[HTML]{333333} 1} \\ \hline
        \end{tabular}
        \end{table}
    \item \textbf{\textcolor{red}{NOT}}
    
    \begin{circuitikz} \draw
        (0,1) node[not port] (mynot1) {}
            (mynot1.in 1) node [anchor=east] {A}
            (mynot1.out)  node [anchor=west] {out};

        \end{circuitikz}
    \begin{table}[H]
        \centering
        \begin{tabular}{|c|
        >{\columncolor[HTML]{F8FF00}}l |}
        \hline
        \multicolumn{1}{|c|}{\cellcolor[HTML]{34CDF9}A} & \multicolumn{1}{c|}{\cellcolor[HTML]{F8FF00}out} \\ \hline
        {\color[HTML]{FE0000} 0}                        & {\color[HTML]{329A9D} 1}                         \\ \hline
        {\color[HTML]{329A9D} 1}                        & 0                                                \\ \hline
        \end{tabular}
        \end{table}

    \item \textbf{\textcolor{red}{NAND}}
    
    \begin{circuitikz} \draw
        (0,1) node[nand port] (mynand1) {}
            (mynand1.in 1) node [anchor=east] {A}
            (mynand1.in 2) node [anchor=east] {B}
            (mynand1.out)  node [anchor=west] {out};

        \end{circuitikz}
    \begin{table}[H]
        \centering
        \begin{tabular}{|l|l|
        >{\columncolor[HTML]{F8FF00}}l |}
        \hline
        \cellcolor[HTML]{34CDF9}A & \cellcolor[HTML]{34CDF9}B & out                      \\ \hline
        0                         & 0                         & 1                        \\ \hline
        0                         & 1                         & 1                        \\ \hline
        1                         & 0                         & 1                        \\ \hline
        {\color[HTML]{FE0000} 1}  & {\color[HTML]{FE0000} 1}  & {\color[HTML]{FE0000} 0} \\ \hline
        \end{tabular}
        \end{table}

    \item \textbf{\textcolor{red}{NOR}}
    
    \begin{circuitikz} \draw
        (0,1) node[nor port] (mynor1) {}
            (mynor1.in 1) node [anchor=east] {A}
            (mynor1.in 2) node [anchor=east] {B}
            (mynor1.out)  node [anchor=west] {out};
    
        \end{circuitikz}
    \begin{table}[H]
        \centering
        \begin{tabular}{|l|l|
        >{\columncolor[HTML]{F8FF00}}l |}
        \hline
        \cellcolor[HTML]{34CDF9}A & \cellcolor[HTML]{34CDF9}B & out                      \\ \hline
        {\color[HTML]{FE0000} 0}  & {\color[HTML]{FE0000} 0}  & {\color[HTML]{FE0000} 1} \\ \hline
        0                         & 1                         & 0                        \\ \hline
        1                         & 0                         & 0                        \\ \hline
        {\color[HTML]{333333} 1}  & {\color[HTML]{333333} 1}  & {\color[HTML]{333333} 0} \\ \hline
        \end{tabular}
        \end{table}

    \item \textbf{\textcolor{red}{XOR}}
    
    \begin{circuitikz} \draw
        (0,1) node[xor port] (myxor1) {}
            (myxor1.in 1) node [anchor=east] {A}
            (myxor1.in 2) node [anchor=east] {B}
            (myxor1.out)  node [anchor=west] {out};
    
        \end{circuitikz}
    \begin{table}[H]
        \centering
        \begin{tabular}{|l|l|
        >{\columncolor[HTML]{F8FF00}}l |}
        \hline
        \cellcolor[HTML]{34CDF9}A & \cellcolor[HTML]{34CDF9}B & out                      \\ \hline
        {\color[HTML]{333333} 0}  & {\color[HTML]{333333} 0}  & {\color[HTML]{333333} 0} \\ \hline
        {\color[HTML]{FE0000} 0}  & {\color[HTML]{FE0000} 1}  & {\color[HTML]{FE0000} 1} \\ \hline
        {\color[HTML]{FE0000} 1}  & {\color[HTML]{FE0000} 0}  & {\color[HTML]{FE0000} 1} \\ \hline
        {\color[HTML]{333333} 1}  & {\color[HTML]{333333} 1}  & {\color[HTML]{333333} 0} \\ \hline
        \end{tabular}
        \end{table}
\end{itemize}

\subsection{Một số đẳng thức cơ bản}

\section{Mạch tổ hợp}

\subsection{Khái niệm}
\begin{itemize}
    \item {Gồm \(n\) ngõ vào (input); \(m\) ngõ ra (output)}
        \SubItem {Mỗi ngõ ra là 1 hàm luận lý của các ngõ vào}
    \item {Mạch tổ hợp không mang tính ghi nhớ: Ngõ ra chỉ phụ thuộc vào Ngõ vào hiện tại, không xét những giá trị trong quá khứ}


\end{itemize}

\subsection{Độ trễ mạch}
\begin{itemize}
    \item Độ trễ mạch (\textcolor{red}{Propagation delay/ gate delay}) = Thời gian điểm tín hiệu ra ổn định \(-\) thời điểm tín hiệu vào ổn định.
    \item Mục tiêu thiết kế mạch: làm giảm thời gian độ trễ mạch.
\end{itemize}

\subsection{Các bước thiết kế}

Thường trải qua 3 bước:
\begin{itemize}
    \item \textcolor{red}{\textbf{Bước 1:} Lập bảng chân trị:}
    \begin{table}[H]
        \centering
        \begin{tabular}{|l|l|
        >{\columncolor[HTML]{F8FF00}}l |}
        \hline
        \cellcolor[HTML]{34CDF9}A & \cellcolor[HTML]{34CDF9}B & F                        \\ \hline
        {\color[HTML]{333333} 0}  & {\color[HTML]{333333} 0}  & {\color[HTML]{333333} 1} \\ \hline
        {\color[HTML]{333333} 0}  & {\color[HTML]{333333} 1}  & {\color[HTML]{333333} 1} \\ \hline
        {\color[HTML]{333333} 1}  & {\color[HTML]{333333} 0}  & {\color[HTML]{333333} 1} \\ \hline
        {\color[HTML]{FE0000} 1}  & {\color[HTML]{FE0000} 1}  & {\color[HTML]{FE0000} 0} \\ \hline
        \end{tabular}
        \end{table}
    \item \textcolor{red}{\textbf{Bước 2:} Viết hàm luận lý} 
    
    \begin{equation*}
        F = \overline{AB}
    \end{equation*}
    
    \item \textcolor{red}{\textbf{Bước 3:} Vẽ sơ đồ mạch và thử nghiệm}
    
    \centering
    \begin{circuitikz} \draw
        (0,1) node[nand port] (mynand1) {}
            (mynand1.in 1) node [anchor=east] {A}
            (mynand1.in 2) node [anchor=east] {B}
            (mynand1.out)  node [anchor=west] {out};
    
        \end{circuitikz}
\end{itemize}

\subsection{Ví dụ}
\begin{tcolorbox}
    \textbf{Ví dụ 1:} Thiết kế mạch cộng 2 bits không nhớ 
\end{tcolorbox}

\begin{itemize}
    \item \textbf{Bước 1:} Lập bảng chân trị
    \begin{table}[H]
        \centering
        \begin{tabular}{|l|l|
        >{\columncolor[HTML]{F8FF00}}l |}
        \hline
        \cellcolor[HTML]{34CDF9}A & \cellcolor[HTML]{34CDF9}B & F                        \\ \hline
        {\color[HTML]{333333} 0}  & {\color[HTML]{333333} 0}  & {\color[HTML]{333333} 0} \\ \hline
        {\color[HTML]{333333} 0}  & {\color[HTML]{333333} 1}  & {\color[HTML]{FE0000} 1} \\ \hline
        {\color[HTML]{333333} 1}  & {\color[HTML]{333333} 0}  & {\color[HTML]{FE0000} 1} \\ \hline
        {\color[HTML]{333333} 1}  & {\color[HTML]{333333} 1}  & {\color[HTML]{333333} 0} \\ \hline
        \end{tabular}
        \end{table}
    \item \textbf{Bước 2:} Lập biểu thức
    
    \begin{align*}
        F & = \overline{A}.B + A.\overline{B} \\
          & = A \oplus B
    \end{align*}

    \item \textbf{Bước 3:} Vẽ mạch

    \centering
    \begin{circuitikz} \draw
    
        (0,1) node[xor port] (myxor1) {}
            (myxor1.in 1) node [anchor=east] {A}
            (myxor1.in 2) node [anchor=east] {B}
            (myxor1.out)  node [anchor=west] {out};
        
        \end{circuitikz}
\end{itemize}

\begin{tcolorbox}
    \textbf{Ví dụ 2:} Thiết kế mạch kiểm tra số nguyên không dấu 3 bits có chia hết cho 3
\end{tcolorbox}



\subsection{Một số mạch tổ hợp cơ bản: mạch cộng,...}

\end{document}
